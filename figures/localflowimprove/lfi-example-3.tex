% Circumscribed Parallelepiped
% Author: Axel Pavillet
\documentclass[tikz,border=0pt]{standalone}
\usepackage{tkz-graph}
\usepackage{relsize}
\usepackage{adjustbox}
\usetikzlibrary{fit,arrows,decorations.markings,calc,shapes,backgrounds}

\definecolor{cd40000}{RGB}{212,0,0}
\definecolor{cffffff}{RGB}{255,255,255}
\definecolor{c2a7fff}{RGB}{42,127,255}
\definecolor{c0000d9}{RGB}{0,0,217}
\definecolor{c0000ff}{RGB}{0,0,255}
%\definecolor{bluecolor}{RGB}{71,190,255}
%\definecolor{bluecolor}{RGB}{255,186,81}
\definecolor{bluecolor}{RGB}{136,224,223}
\definecolor{darkgrey}{RGB}{150,150,150}
\definecolor{lightgrey}{RGB}{226,226,226}
​
%\definecolor{bluecolor}{RGB}{25,174,255}
\colorlet{Rcolor}{bluecolor}
​
%\definecolor{bordercolor}{RGB}{25,174,255}
\colorlet{bordercolor}{bluecolor}
​
%\definecolor{Rbarcolor}{RGB}{25,174,255}
%\definecolor{Rbarcolor}{RGB}{226,226,226}
\colorlet{Rbarcolor}{lightgrey}
​
\colorlet{scolor}{bluecolor}
\colorlet{tcolor}{bluecolor}
​
%\definecolor{scolor}{RGB}{25,174,255}
%\definecolor{tcolor}{RGB}{25,174,255}
​
\usepackage{relsize}
​
% \tikzset{every path/.style={draw, ->,>=latex,line width=1pt,color=black}}
​
%\tikzset{every node/.style={draw,minimum width = 20pt,line width = 1.75pt,
%circle,inner sep=0pt,font = \fontsize{14}{14}\selectfont,fill=none}}
\newcommand{\n}{\text{\raisebox{0.3ex}{-}}}
​
\begin{document}
​
\tikzset{res/.style={ellipse,draw,minimum height=0.5cm,minimum width=0.8cm}}
\tikzset{hfit/.style={rounded rectangle, inner xsep=0pt, fill=#1!30},
           vfit/.style={rounded corners, fill=#1!30}}

\begin{tikzpicture}[xscale=0.45,yscale=0.4,myellipse/.style={
          fit=(a.corner 2) (a.corner 3),
          draw, ellipse,
          inner xsep=0mm,
          inner ysep=3mm}]]
​
% \draw[color=black,line width=1pt] (0,0) rectangle ++(15,5);
​
\draw[color=white,rounded corners=10pt,line width=0pt,fill=Rcolor] (3,-0.15) rectangle ++(4.5,4.8);
​
\draw[color=white,rounded corners=10pt,line width=0pt,fill=Rbarcolor] (8.8,-0.15) rectangle ++(4.5,4.8);
​
%\draw [rotate around={-10:(10.95,3.85)}] (10.95,3.85) ellipse (0.7in and 0.2in);
%\draw [rotate around={-70:(9.75,3.275)},line width=0.0pt,color=bordercolor,fill=bordercolor] (9.75,3.275) ellipse (0.47in and 0.2in);
%\draw [rotate around={30:(9.5,4.1)},line width=0.0pt,color=bordercolor,fill=bordercolor] (9.5,4.1) ellipse (0.25in and 0.2in);
%\draw [line width=0.5cm, color = Rcolor, line cap = round] (9.5,4.1) -- (9.5,4.1);
\draw [line width=0.5pt, double distance=0.5cm, double=Rcolor, color = orange, line cap = round, stroke=black] (9.5,4.1) -- (9.5,4.1);

%\draw [line width=0.5cm, color = Rcolor, line cap = round] (12,3.5) -- (10,2.7);
\draw [line width=0.5pt, double distance=0.5cm, double=Rcolor, color = orange, line cap = round, stroke=black] (12,3.5) -- (10,2.7);

%\draw [rotate around={25:(11,3.1)},line width=0.0pt,color=bordercolor,fill=bordercolor] (11,3.1) ellipse (0.6in and 0.3in);
​
\begin{scope}[every node/.style={draw,minimum size=15,line width=1pt,circle,fill=none,font=\fontsize{14}{14}\selectfont,inner sep=0pt}]
\node[fill=scolor,text=black] (s) at (0.75,2.5) {$s$};
\node[fill=tcolor,text=black] (t) at (14.5,2.5) {$t$};
\end{scope}
​
​
\begin{scope}[every node/.style={draw,minimum size=10,line width=0.75pt,text=Rcolor,circle,fill=none,font=\fontsize{5}{5}\selectfont,inner sep=0pt}]
\node (R1) at (5,4.2) {R1};
\node (R2) at (6,3.3) {R2};
\node (R3) at (3.5,2.8) {R3};
\node (R4) at (6.5,1.8) {R4};
\node (R5) at (5.5,0.4) {R5};
%\node (R6) at (3,1.5) {R6};
\node (aR1) at (9.5,4.1) {a1};
\node (aR2) at (12,3.5) {a2};
\node (aR3) at (10,2.7) {a3};
\node [color=darkgrey,text=lightgrey] (U1) at (10.3,1.6) {U1};
\node [color=darkgrey,text=lightgrey] (U2) at (12.1,0.3) {U2};
\node [color=darkgrey,text=lightgrey] (U3) at (10.5,0.4) {U3};
\node [color=darkgrey,text=lightgrey] (U4) at (12.2,2.25) {U4};
\end{scope}
%
\begin{scope}[line width=1pt,color=black]
\draw[-] (R1) to (R2);
\draw[-] (R2) to (R4);
\draw[-] (R2) to (R5);
\draw[-] (R4) to (R5);
\draw[-] (R5) to (R3);
\draw[-] (R3) to (R1);
\draw[-] (R3) to (R4);
\draw[-] (R3) to (R2);
% \draw[-] (R6) to (R3);
% \draw[-] (R6) to (R4);
% \draw[-] (R6) to (R5);
\draw[-] (aR1) to (aR2);
\draw[-] (aR1) to (aR3);
\draw[-] (aR2) to (aR3);
\draw[-,color=darkgrey] (U1) to (U2);
\draw[-,color=darkgrey] (U1) to (U3);
\draw[-,color=darkgrey] (U1) to (U4);
\draw[-,color=darkgrey] (U2) to (U3);
\draw[-,color=darkgrey] (U2) to (U4);
\draw[-,color=darkgrey] (U3) to (U4);
\draw[-] (R1) to (aR1);
\draw[-] (R2) to (aR3);
\draw[-,color=darkgrey] (aR2) to (U4);
\draw[-,color=darkgrey] (aR3) to (U1);
\end{scope}
​
%
\begin{scope}[line width=1pt]
\draw[-] (s) to [bend left=20] (R1);
\draw[-] (s) to [bend left=15] (R2);
\draw[-] (s) to [bend right=0] (R3);
\draw[-] (s) to [bend right=10] (R4);
\draw[-] (s) to [bend right=20] (R5);
\end{scope}
​
\begin{scope}[line width=1pt]
\draw[-] (aR1) to [bend left=30] (t);
\draw[-] (aR2) to [bend left=0] (t);
\draw[-] (aR3) to [bend right=-10] (t);
\end{scope}
​
\begin{scope}[line width=1pt]
\draw[-,color=darkgrey] (U1) to [bend right=15] (t);
\draw[-,color=darkgrey] (U2) to [bend right=15] (t);
\draw[-,color=darkgrey] (U3) to [bend right=10] (t);
\draw[-,color=darkgrey] (U4) to [bend left=0] (t);
\end{scope}
​
% \draw[color=cd40000,rounded corners=10pt,line width=2pt,dashed] (2.25,-0.8) rectangle ++(4.75,10);
% \draw[color=c2a7fff,rounded corners=10pt,line width=2pt,dashed] (8,-0.8) rectangle ++(4.75,10);
% \draw[color=yellow,rounded corners=10pt,line width=2pt,dashed] (8.25,4.5) rectangle ++(4.25,4.25);
​
\node[line width=0,font=\fontsize{11}{14}\selectfont] at (6.7,0.5) {$R$};
\node[line width=0,font=\fontsize{11}{14}\selectfont] at (9.5,0.5) {$\bar{R}$};
\node[line width=0,font=\fontsize{8}{8}\selectfont] at (8.9,4.5) {$B$};
\node[line width=0,font=\fontsize{8}{8}\selectfont] at (8.7,2) {$\partial{(\!R\!\cup\!B\!)}$};

\end{tikzpicture}
​
\end{document}
